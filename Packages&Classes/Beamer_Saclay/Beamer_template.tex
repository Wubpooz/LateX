\documentclass[usenames,dvipsnames,aspectratio=169]{beamer} %for \cul xcolor tags to work with tikz, passes args to xcolor directly => DO NOT CALL the package with them later else causes error

\usepackage{Mmbase}

\usetheme{psaclay}

\makeatletter
\newcommand{\Pause}[1][]{\unless\ifmeasuring@\relax
\pause[#1]%
\fi}
\makeatother %for correct \pause in math mode in enumerate


%------------------------------ Title page ------------------------------

\title[Intro Fourier]{Introduction à la théorie de Fourier}

\subtitle{Généralités}

\author{M. Waharte}

\institute[PPS]{Polytech Paris-Saclay}

\date[MAT342]{MAT 342 : Théorie de Fourier - 2022}


%------------------------------------------------------------
%beginning of each section and highlights the current section

\AtBeginSection[]{
  \begin{frame}
    \frametitle{Sommaire}
    \tableofcontents[currentsection]
  \end{frame}
}
%------------------------------------------------------------


\begin{document}
\begin{frame}
    \titlepage
\end{frame}


% 1.----------------------------- ToC ----------------------------

\begin{frame}
\frametitle{Sommaire}
\tableofcontents
\end{frame}


% 2.---------------------------------------------------------

\section{Définition}


%3. ---------------------------------------------------------

\begin{frame}
\frametitle{Définition}

\begin{block}{Définition}
Soit $f : \R^n \ra \C$ intégrable, on note sa transformée de Fourier $\hat{f}$ ( ou $\fF(f)$) :  $$\hat{f} : \xi\in\R^n \mapsto \int_\R f(x) e^{-2i\pi x\xi} \dx$$
\end{block}
\pause

\begin{itemize} %<1-> means 1st slide and after and <2> only 2nd slide
    \item<1-> On peut aussi le définir :\\ • $\hat{f}(\xi)\mapsto \int_\R f(x) e^{-x\xi}\dx$\\\pause
    \uncover<2->{ • $\hat{f}(\xi)\mapsto \frac{1}{(\sqrt{2\pi})^n}\int_\R f(x) e^{-x\xi}\dx$}
\end{itemize}
\uncover<2->{(La seule différence sera les coefficient des formules)}

\end{frame}


%4. ---------------------------------------------------------


\begin{frame}
\frametitle{Exemple}
\begin{block}{Exemple}
Calculez la transformée de Fourier de $\mathbbm{1}_{[a,b]}$ et de $x\mapsto e^{-|x|}$.\\
\end{block} \pause

\begin{equation*}
   \begin{split} 
        \hat{f}(\xi) &= \int_\R e^{-|x|} e^{-2i\pi x\xi}\,dx = \Pause \int_{\R^-} e^{x(1-2i\pi\xi)}dx\; + \; \int_{\R^+} e^{x(-1-2i\pi\xi)}dx \\
        &\cul[OliveGreen]{ = \frac{2}{1+4\pi\xi^2} }\,.
    \end{split}  
\end{equation*}

\end{frame}


%5. ---------------------------------------------------------

\section{Propriétés}


%6. ---------------------------------------------------------

\begin{frame}
\frametitle{Propriétés}

Soient la translation $\tau_y(f) : x\in\R^n \mapsto f(x-y)$ et $\sC_{\ra0}$ l'ensemble des fonctions continues convergeant vers $0$ en $\pm\,\infty$.

\begin{block}{Propriétés}
Soient $f,g\in L^1(\R^n)$, $\lambda\in\R$ et $k\in\N$.

\begin{enumerate}
    \item Si $h : x\mapsto e^{2\pi i \lambda x}f(x)$, alors $h$ intégrable et $\hat{h}=\tau_\lambda f$.\pause
    \item $\fF(\tau_\lambda f)(\xi) = e^{-2i\pi x\xi} \hat{f}(\xi)$.\pause
    \item $\fF (f \ast g) = \hat{f}\hat{g}$.\pause
    \item Si $\forall \alpha\in\N^n\; |\alpha|\leq k, (x\mapsto x^\alpha f(x))\in L^1(\R)$, alors $\hat{f}\in\sC^k$ et $\partial^\alpha f = \fF[(-2i\pi x)^\alpha f(x)]$.\pause
    \item Si $f\in\sC^k$ et $\forall |\alpha|\leq k, \;\partial^\alpha f\in L^1$ et $\in \sC_{\ra 0}$ avec $k-1$, alors $\fF(\partial^\alpha f)(\xi) = (2i\pi\xi)^\alpha \hat{f}(\xi)$.
\end{enumerate}

\end{block}

\end{frame}

%7. ----------------------------------------------------

\begin{frame}{Remarque}
Une fonction est dite à décroisssance rapide si $\forall n, \; \lim_{x\ra\pm\infty} x^n f(x) = 0$ ; la transformée de Fourier d'une telle fonction sera $\sC^\infty$ et réciproquement.\\

\pause
\begin{block}{Théorème de Riemann-Lebesgue}
$\fF(L^1)\subset L^\infty \cap \sC_{\ra 0}$.\\
i.e. pour $f\in L^1$, $\hat{f}$ est continue, bornée et converge vers $0$ en $\pm\infty$ (la TF est régulière).

\end{block}


\end{frame}


%8. ----------------------------------------------------

\section{Transformée inverse}

%9. ----------------------------------------------------


\begin{frame}{Définition}
   \begin{block}{Définition (Transformée inverse)}
        Soit $g\in L^1$, $$ \fF^{-1}(g) = \check{g} : x\mapsto \int_\R g(\xi) e^{2i\pi x\xi}d\xi \ = \hat g(-x)\;.$$
   \end{block}
   \pause
   \underline{Théorème d'inversion :} Soit $f\in L^1$ telle que $\hat{f}$ intégrable alors, $\hat{f}$ est continue et $|| f- \fF^{-1}(\hat{f})||_1 = 0$.\\ 
   \pause
   De plus, si $f$ continue, $f = \fF^{-1}(\hat{f})$.\\
   $\implies$ $\fF$ est injective sur $L^1$.
    
\end{frame}


%10. ----------------------------------------------------

\begin{frame}{Espace de Schwarz}
    \begin{block}{Définition (Espace de Schwarz)}
        On note $\mS$, espace de Schwarz, l'ensemble des fonctions infiniment dérivables telles que $$\forall \alpha,\beta \in \N^n,\; \lim_{|x|\ra\infty} x^\alpha \partial^\beta =0\;.$$
    \end{block}
    \pause
    On y retrouve notamment $x\mapsto e^{-|x|^2}$ et les fonctions $\sC^\infty$ à support compact.\\
    \text{ }\\
    \pause
    $\fF : \mS\ra \mS$.
\end{frame}

%11. -------------------------------------------------------

\begin{frame}{Plancheret-Parseval}
    
    \begin{block}{Théorème de Plancheret-Parseval (simplifié)}
        Si $f\in L^1\cap L^2$, alors $\hat{f}\in L^2$ et $||f||_2 = ||\hat{f}||_2$.
    \end{block}

\end{frame}

%12. ---------------------------------------------------------

\begin{frame}[fragile]
The following Python code adds two numbers and display the result using \verb|print()| function
\begin{minted}{C}
    #include <stdio.h>

    int main(int argc, char* argv[]){
        return 0;
    }
\end{minted}
\end{frame}

\end{document}