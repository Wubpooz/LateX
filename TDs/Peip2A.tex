\documentclass{article}
\usepackage[utf8]{inputenc}
\usepackage[fleqn]{amsmath}
\usepackage{amssymb}
\usepackage[top=2cm, bottom=2cm, left=2.6cm, right=2.6cm]{geometry}
\usepackage{setspace}
\usepackage{eucal}
\usepackage{stmaryrd}
\usepackage{mathrsfs}
\usepackage{mathtools}
\usepackage{fourier-orns}
\usepackage{centernot}
\usepackage{multirow}
\usepackage{hyperref}
\usepackage{fancyhdr}
\usepackage{xcolor}


\everymath{\displaystyle}

\newcommand{\mA}{\mathcal{A}}
\newcommand{\mB}{\mathcal{B}}
\newcommand{\mC}{\mathcal{C}}
\newcommand{\mD}{\mathcal{D}}
\newcommand{\mE}{\mathcal{E}}
\newcommand{\mF}{\mathcal{F}}
\newcommand{\mG}{\mathcal{G}}
\newcommand{\mH}{\mathcal{H}}
\newcommand{\mI}{\mathcal{I}}
\newcommand{\mJ}{\mathcal{J}}
\newcommand{\mK}{\mathcal{K}}
\newcommand{\mL}{\mathcal{L}}
\newcommand{\mM}{\mathcal{M}}
\newcommand{\mN}{\mathcal{N}}
\newcommand{\mO}{\mathcal{O}}
\newcommand{\mP}{\mathcal{P}}
\newcommand{\mQ}{\mathcal{Q}}
\newcommand{\mR}{\mathcal{R}}
\newcommand{\mS}{\mathcal{S}}
\newcommand{\mT}{\mathcal{T}}
\newcommand{\mU}{\mathcal{U}}
\newcommand{\mV}{\mathcal{V}}
\newcommand{\mW}{\mathcal{W}}
\newcommand{\mX}{\mathcal{X}}
\newcommand{\mY}{\mathcal{Y}}
\newcommand{\mZ}{\mathcal{Z}}

\newcommand{\R}{\mathbb{R}}
\newcommand{\N}{\mathbb{N}}
\newcommand{\C}{\mathbb{C}}
\newcommand{\Z}{\mathbb{Z}}
\newcommand{\U}{\mathbb{U}}
\newcommand{\D}{\mathbb{D}}
\newcommand{\K}{\mathbb{K}}
\newcommand{\bP}{\mathbb{P}}
\newcommand{\E}{\mathbb{E}}
\newcommand{\V}{\mathbb{V}}

\newcommand{\sA}{\mathscr{A}}
\newcommand{\sB}{\mathscr{B}}
\newcommand{\sC}{\mathscr{C}}
\newcommand{\sD}{\mathscr{D}}
\newcommand{\sE}{\mathscr{E}}
\newcommand{\sF}{\mathscr{F}}
\newcommand{\sG}{\mathscr{G}}
\newcommand{\sH}{\mathscr{H}}
\newcommand{\sI}{\mathscr{I}}
\newcommand{\sJ}{\mathscr{J}}
\newcommand{\sK}{\mathscr{K}}
\newcommand{\sL}{\mathscr{L}}
\newcommand{\sM}{\mathscr{M}}
\newcommand{\sN}{\mathscr{N}}
\newcommand{\sO}{\mathscr{O}}
\newcommand{\sP}{\mathscr{P}}
\newcommand{\sQ}{\mathscr{Q}}
\newcommand{\sR}{\mathscr{R}}
\newcommand{\sS}{\mathscr{S}}
\newcommand{\sT}{\mathscr{T}}
\newcommand{\sU}{\mathscr{U}}
\newcommand{\sV}{\mathscr{V}}
\newcommand{\sW}{\mathscr{W}}
\newcommand{\sX}{\mathscr{X}}
\newcommand{\sY}{\mathscr{Y}}
\newcommand{\sZ}{\mathscr{Z}}

\newcommand{\Mat}{\text{Mat}}

\newcommand{\Ra}{\Rightarrow}
\newcommand{\ra}{\rightarrow}
\newcommand{\La}{\Leftarrow}
\newcommand{\la}{\leftarrow}
\newcommand{\Da}{\Leftrightarrow}
\newcommand{\da}{\leftrightarrow}

\newcommand{\0}{\emptyset}
\newcommand{\bigzero}{\mbox{\normalfont\Large\bfseries 0}}
\newcommand{\rvline}{\hspace*{-\arraycolsep}\vline\hspace*{-\arraycolsep}}
\newcommand\restr[2]{{% we make the whole thing an ordinary symbol
  \left.\kern-\nulldelimiterspace % automatically resize the bar with \right
  #1 % the function
  \vphantom{\big|} % pretend it's a little taller at normal size
  \right|_{#2} % this is the delimiter
  }}

\newcommand{\SN}{\sum_{i=1}^n}
\newcommand{\SI}{\sum_{n=0}^{+\infty}}
\newcommand{\PN}{\prod_{i=1}^n}
\newcommand{\TI}{\xrightarrow[+\infty]{}}
\newcommand{\LI}{\lim_{n\to +\infty}}
\newcommand{\dt}{\; \text{d}t}
\newcommand{\dx}{\; \text{d}x}

\renewcommand{\headrulewidth}{1pt}
\renewcommand{\footrulewidth}{1pt}


\pagestyle{fancy}
\fancyhf{}
\lhead{Polytech Paris-Saclay}
\chead{\textbf{Feuille d'exercices n°1 Équivalents, Suites et DL}}
\rhead{Peip2A}
\lfoot{Mathieu WAHARTE}
\rfoot{\href{mailto:mathieu.waharte@universite-pais-saclay.fr}{mathieu.waharte@u-psud.fr}}
\cfoot{\thepage}


\usepackage{titlesec}

\titleformat{\subsection}[runin]
{\normalfont\large\bfseries}{\thesubsection}{1em}{}

\begin{document}

\section*{Équivalents}
\subsection*{Exercice 1.}
\hyperref[subsec:corr1]{(\textit{Voir corrigé})} ($\bigstar$)
\label{subsec:ex1}
\begin{flushleft}
Trouvez des équivalents des suites et fonctions suivantes :

\begin{tabular}{l l l}
     1) $u_n = \frac{1}{n-1} - \frac{1}{n+1}$ en $+\infty$ & 2) $v_n = \sqrt{n+1} - \sqrt{n-1}$ en $+\infty$ & 3) $w_n =\sqrt{n + \sqrt{n}} - \sqrt{n}$ en $+\infty$\\
     4) $\ln(\cos x)$ en $0$ & 5) $\frac{1}{x} - \frac{1}{\sin x}$ en $0$ & 6) $\cos(\sin x)$ en $0$ \\
\end{tabular}

\end{flushleft}
\subsection*{Exercice 2.}
\hyperref[subsec:corr2]{(\textit{Voir corrigé})} ($\bigstar$)
\label{subsec:ex2}
\begin{flushleft}
Comparez les fonctions suivantes :\\
1) $x \ln x$ et $\ln(1+2x)$ au voisinage de 0\\
2) $x \ln x$ et $\sqrt{x^2 + 3x} \ln(x^2) \sin x$ au voisinage de $+\infty$\\
\end{flushleft}


\section*{Développements limités}
\subsection*{Exercice 3}
\hyperref[subsec:corr3]{(\textit{Voir énoncé})} ($\bigstar$)
\label{subsec:ex3}
\begin{flushleft}
1)
\end{flushleft}


\newpage

\section*{Équivalents}
\subsection*{Exercice 1.}
\hyperref[subsec:ex1]{(\textit{Voir corrigé})} ($\bigstar$)
\label{subsec:corr1}
\begin{flushleft}
1) On met sur le même dénominateur.\\
2) On utilise la quantité conjugée : $v_n = \frac{(\sqrt{n+1} + \sqrt{n-1})(\sqrt{n+1}-\sqrt{n-1})}{\sqrt{n+1 + \sqrt{n-1}}} = \dots = \frac{2}{\sqrt{n+1}+\sqrt{n-1}}$.\\
Aussi, on a $\frac{\sqrt{n+1}+\sqrt{n-1}}{2\sqrt{n}} = \frac{1}{2}\sqrt{1+\frac{1}{n}} + \frac{1}{2}\sqrt{1-\frac{1}{n}}$, ces deux quantités tendant vers 1, on a un équivalent (avec les $0.5$).
Ainsi $v_n \underset{+\infty}{\sim} = \frac{1}{\sqrt{n}}$.\\
3) $w_n = \sqrt{n\Big(1+\frac{1}{\sqrt{n}}\Big)} - \sqrt{n} = \sqrt{n}\Big(\sqrt{1+\frac{1}{n}} -1\Big) \underset{+\infty}{\sim} $\\
4) $\ln(\cos x) = \ln(1 - (1- \cos x)) \underset{0}{\sim} - (1-\cos x) \underset{0}{\sim} - \frac{x^2}{2}$.\\
5) $\frac{1}{x}-\frac{1}{\sin x} = \frac{\sin x -x}{x\sin x} \underset{0}{\sim} \frac{\frac{-x^3}{3!}}{x^2} = -\frac{x}{6}$ (DL ordre 4 de sin en haut et ordre 2 en bas).\\
\text{ }\\
6) Par composition, $\cos (\sin x) \ra 1$ en 0 et donc $\cos (\sin x) \underset{0}{\sim} 1$.\\
\end{flushleft}

\subsection*{Exercice 2.}
\hyperref[subsec:ex2]{(\textit{Voir énoncé})} ($\bigstar$)
\label{subsec:corr2}
\begin{flushleft}
1) $\frac{\ln(1+2x)}{x\ln x} \underset{0}{\sim} \frac{2}{\ln x} \longrightarrow 0 $.\\
2) Pour $x$ assez grand, $\sqrt{x^2 + 3x}\leqslant 2x$ (passer au carré pour s'en convaincre) et on a aussi $\ln(x^2) = 2 \ln x$ et $| \sin x | \leqslant 1$.\\
Donc $|\sqrt{x^2 + 3x} \ln(x^2) \sin x |\leqslant 4x\ln x$ et ainsi $\sqrt{x^2 + 3x} \ln(x^2) \sin x \underset{+\infty}{=} O(x \ln x)$.\\
\end{flushleft}

\section*{Dévelopments limités}
\subsection*{Exercice 3}
\hyperref[subsec:ex3]{(\textit{Voir énoncé})} ($\bigstar$)
\label{subsec:corr3}
\begin{flushleft}
1)
\end{flushleft}

\end{document}
