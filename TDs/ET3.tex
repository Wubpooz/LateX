\documentclass{article}
\usepackage[utf8]{inputenc}
\usepackage[fleqn]{amsmath}
\usepackage{amssymb}
\usepackage[top=2cm, bottom=2cm, left=2.6cm, right=2.6cm]{geometry}
\usepackage{setspace}
\usepackage{eucal}
\usepackage{stmaryrd}
\usepackage{mathrsfs}
\usepackage{mathtools}
\usepackage{fourier-orns}
\usepackage{centernot}
\usepackage{multirow}
\usepackage{hyperref}
\usepackage{fancyhdr}
\usepackage{xcolor}
\usepackage{titlesec}
\usepackage[bottom]{footmisc}


\newcommand{\mA}{\mathcal{A}}
\newcommand{\mB}{\mathcal{B}}
\newcommand{\mC}{\mathcal{C}}
\newcommand{\mD}{\mathcal{D}}
\newcommand{\mE}{\mathcal{E}}
\newcommand{\mF}{\mathcal{F}}
\newcommand{\mG}{\mathcal{G}}
\newcommand{\mH}{\mathcal{H}}
\newcommand{\mI}{\mathcal{I}}
\newcommand{\mJ}{\mathcal{J}}
\newcommand{\mK}{\mathcal{K}}
\newcommand{\mL}{\mathcal{L}}
\newcommand{\mM}{\mathcal{M}}
\newcommand{\mN}{\mathcal{N}}
\newcommand{\mO}{\mathcal{O}}
\newcommand{\mP}{\mathcal{P}}
\newcommand{\mQ}{\mathcal{Q}}
\newcommand{\mR}{\mathcal{R}}
\newcommand{\mS}{\mathcal{S}}
\newcommand{\mT}{\mathcal{T}}
\newcommand{\mU}{\mathcal{U}}
\newcommand{\mV}{\mathcal{V}}
\newcommand{\mW}{\mathcal{W}}
\newcommand{\mX}{\mathcal{X}}
\newcommand{\mY}{\mathcal{Y}}
\newcommand{\mZ}{\mathcal{Z}}

\newcommand{\R}{\mathbb{R}}
\newcommand{\N}{\mathbb{N}}
\newcommand{\C}{\mathbb{C}}
\newcommand{\Z}{\mathbb{Z}}
\newcommand{\U}{\mathbb{U}}
\newcommand{\D}{\mathbb{D}}
\newcommand{\K}{\mathbb{K}}
\newcommand{\bP}{\mathbb{P}}
\newcommand{\E}{\mathbb{E}}
\newcommand{\V}{\mathbb{V}}

\newcommand{\sA}{\mathscr{A}}
\newcommand{\sB}{\mathscr{B}}
\newcommand{\sC}{\mathscr{C}}
\newcommand{\sD}{\mathscr{D}}
\newcommand{\sE}{\mathscr{E}}
\newcommand{\sF}{\mathscr{F}}
\newcommand{\sG}{\mathscr{G}}
\newcommand{\sH}{\mathscr{H}}
\newcommand{\sI}{\mathscr{I}}
\newcommand{\sJ}{\mathscr{J}}
\newcommand{\sK}{\mathscr{K}}
\newcommand{\sL}{\mathscr{L}}
\newcommand{\sM}{\mathscr{M}}
\newcommand{\sN}{\mathscr{N}}
\newcommand{\sO}{\mathscr{O}}
\newcommand{\sP}{\mathscr{P}}
\newcommand{\sQ}{\mathscr{Q}}
\newcommand{\sR}{\mathscr{R}}
\newcommand{\sS}{\mathscr{S}}
\newcommand{\sT}{\mathscr{T}}
\newcommand{\sU}{\mathscr{U}}
\newcommand{\sV}{\mathscr{V}}
\newcommand{\sW}{\mathscr{W}}
\newcommand{\sX}{\mathscr{X}}
\newcommand{\sY}{\mathscr{Y}}
\newcommand{\sZ}{\mathscr{Z}}

\newcommand{\Mat}{\text{Mat}}

\newcommand{\Ra}{\Rightarrow}
\newcommand{\ra}{\rightarrow}
\newcommand{\La}{\Leftarrow}
\newcommand{\la}{\leftarrow}
\newcommand{\Da}{\Leftrightarrow}
\newcommand{\da}{\leftrightarrow}

\newcommand{\0}{\emptyset}
\newcommand{\bigzero}{\mbox{\normalfont\Large\bfseries 0}}
\newcommand{\rvline}{\hspace*{-\arraycolsep}\vline\hspace*{-\arraycolsep}}
\newcommand\restr[2]{{% we make the whole thing an ordinary symbol
  \left.\kern-\nulldelimiterspace % automatically resize the bar with \right
  #1 % the function
  \vphantom{\big|} % pretend it's a little taller at normal size
  \right|_{#2} % this is the delimiter
  }}

\newcommand{\SN}{\sum_{i=1}^n}
\newcommand{\SI}{\sum_{n=0}^{+\infty}}
\newcommand{\PN}{\prod_{i=1}^n}
\newcommand{\TI}{\xrightarrow[+\infty]{}}
\newcommand{\LI}{\lim_{n\to +\infty}}
\newcommand{\dt}{\; \text{d}t}
\newcommand{\dx}{\; \text{d}x}
\makeatletter
\newcommand\footnoteref[1]{\protected@xdef\@thefnmark{\ref{#1}}\@footnotemark}
\makeatother

\renewcommand{\headrulewidth}{1pt}
\renewcommand{\footrulewidth}{1pt}

\everymath{\displaystyle}
\pagestyle{fancy}
\fancyhf{}

\lhead{Polytech Paris-Saclay}
\chead{\textbf{Feuille d'exercices n°1- Probabilté}}
\rhead{ET3}
\lfoot{Mathieu WAHARTE}
\rfoot{\href{mailto:mathieu.waharte@universite-pais-saclay.fr}{mathieu.waharte@u-psud.fr}}
\cfoot{\thepage}

\titleformat{\subsection}[runin]
{\normalfont\large\bfseries}{\thesubsection}{1em}{}

\begin{document}


\section*{Variables aléatoires réelles}

\subsection*{Exercice 1.}
\hyperref[subsec:corr1]{(\textit{Voir corrigé})} ($\bigstar$)
\label{subsec:ex1}
\begin{flushleft}
1) Soit $X$ une variable aléatoire.\\
a) Rappeler la définition de "$X$ est d'espérance finie ".\\
b) Montrer que $X$ est d'espérance finie si et seulement si $|X|$ est d'espérance fnie.\\
\text{ }\\
2) Soit $X$ variable aléatoire symétrique \footnote{\label{sym} $X$ est symétrique si $P(X=-x)=P(X=x)$} et $f :\R\ra\R$ impaire. Montrer que $f(X)$ est symétrique et $f(X)$ d'espérance finie $\Ra \E(f(X))=0$.\\
\end{flushleft}




\subsection*{Exercice 2.}
\hyperref[subsec:corr2]{(\textit{Voir corrigé})} ($\bigstar \bigstar\clubsuit$ \footnote{ Exercice classique} )
\label{subsec:ex2}
\begin{flushleft}
Soit $X$ suivant la loi de Poisson de paramètre $\lambda$,\\
1) Exprimer le paramètre $p$ de la loi binomiale tel que, à $\lambda$ et $k\in\N$ fixés, Bin$(n,p) \underset{n\to\infty}{\longrightarrow} P(X=k)$.\\
2) En déduire que $\sum_{k\in\N} \Big| \binom{n}{p} p^k (1-p)^{n-k} - e^{-np}\frac{(np)^k}{k!} \Big|\leqslant 2p$.\\
3) Que vient-on de montrer ?\\
\end{flushleft}


\subsection*{Exercice 3.}
\hyperref[subsec:corr3]{(\textit{Voir corrigé})} ($\bigstar \bigstar\clubsuit$)
\label{subsec:ex3}
\begin{flushleft}
Montrer qu'une loi de Poisson de paramètre $\lambda$ converge vers la loi normale de $\mu=\lambda$ et $\sigma = \sqrt{\lambda}$ quand $\lambda\to+\infty$.\\
(En pratique on considérera cette approximation comme bonne pour $\lambda>5$)\\
\end{flushleft}

\subsection*{Exercice 4.}
\hyperref[subsec:corr4]{(\textit{Voir corrigé})}
($\bigstar\bigstar\clubsuit$)
\label{subsec:ex4}
\begin{flushleft}
Marche aléatoire sur un carré :\\
Soit $\Gamma = \{A,B,C,D,O\}$, $ABCD$ un carré de centre $O$ ; Marc se déplace aléatoirement sur $\Gamma$ en sautant d'un point sur un autre, adjacent entres-eux. On considère tous les sauts possibles à une étape comme équiprobables et l'on impose que Marc change de sommet à chaque étape. Á l'étape 0, Marc est en $O$ et on note pour $n\geqslant1$ $O_n$ : "Marc est en $O$ à l'issue du $n$-ième saut" et de même pour $A,B,C,D$. Pour simplifier, on notera $p_n=P(O_n)$, $p_0=1$.\\
1) Calculez $p_1$, $p_2$.\\
2) Montrer $\forall n\in\N^*,\, P(A_n)=P(B_n)=P(C_n)=P(D_n)$.\\
(\textit{Indication : raisonner par récurrence})\\
3) Montrer $\forall n\in\N,\, p_n = \frac{1}{3}(1-p_n)$ et en déduire la valeur explicite de $p_n$.\\
4) Conclure quant au temps passé en chacun des points de $\Gamma$ (on pourra faire un graphique "baton").\\

\end{flushleft}


\section*{Indépendance}
\subsection*{Exercice 5.}
\hyperref[subsec:corr5]{(\textit{Voir corrigé})}
($\bigstar\bigstar$)
\label{subsec:ex5}
\begin{flushleft}
Soit $n>1$ fixé,on choisit de manière équiprobable $x$ parmi $\{1,\dots,n\}$ et l'on note $\forall m\leqslant n$ $A_m$ : “$m$ divise $n$“. On pose aussi $B$ : “$x$ est premier avec $n$“ et on note $p_1,\dots,p_r$ les facteurs premiers de $n$.\\
1) Exprimer $B$ en fonction des $A_{p_k}$.\\
2) $\forall\; m$ tel que $m|n$, calculer $P(A_m)$.\\
3) Montrer que les $A_{p_1},\dots,A_{p_r}$ sont indépendants. En déduire la probabilité de $B$.\\
4) On note $\varphi(n)$, fonction indicatrice d'Euler, le nombre d'entiers entre $1$ et $n$ premiers à $n$. Montrer que $\varphi(n)=n\,\prod_{k=1}^r \Big(1 - \frac{1}{p_k}\Big)$ .\\

\end{flushleft}

\section*{Couples de VA}
\subsection*{Exercice 6.}
\hyperref[subsec:corr6]{(\textit{Voir corrigé})}($\bigstar$)
\label{subsec:ex6}
\begin{flushleft}
Soit $X$, $Y$ deux variables aléatoires symétriques \footnoteref{sym} indépendantes. Montrer que X+Y est symétrique.\\
(\textit{Indication : on pourra comparer la loi de $(-X,-Y)$ à celle de $(X,Y)$})\\
\end{flushleft}

\subsection*{Exercice 7.}
\hyperref[subsec:corr7]{(\textit{Voir corrigé})}($\bigstar$)
\label{subsec:ex7}
\begin{flushleft}
Soit 

\end{flushleft}


\newpage


\subsection*{Exercice 1.}
\hyperref[subsec:ex1]{(\textit{Voir corrigé})} ($\bigstar$)
\label{subsec:corr1}
\begin{flushleft}
1)\\
\end{flushleft}


\subsection*{Exercice 2.}
\hyperref[subsec:ex2]{(\textit{Voir énoncé})} ($\bigstar\bigstar\clubsuit$)
\label{subsec:corr2}
\begin{flushleft}
1)\\

3) On a montré le théorème de Stein qui dit que la loi de Poisson peut approcher la loi binomiale (on a même montré à quel point cette approximation est bonne).\\
\end{flushleft}


\subsection*{Exercice 3.}
\hyperref[subsec:ex3]{(\textit{Voir énoncé})} ($\bigstar\bigstar\clubsuit$)
\label{subsec:corr3}
\begin{flushleft}
1)\\
\end{flushleft}

\subsection*{Exercice 4.}
\hyperref[subsec:ex4]{(\textit{Voir énoncé})}
($\bigstar\bigstar\clubsuit$)
\label{subsec:corr4}
\begin{flushleft}
1)\\
\end{flushleft}

\subsection*{Exercice 5.}
\hyperref[subsec:ex5]{(\textit{Voir énoncé})}
($\bigstar\bigstar$)
\label{subsec:corr5}
\begin{flushleft}
1)\\
\end{flushleft}

\subsection*{Exercice 6.}
\hyperref[subsec:ex6]{(\textit{Voir énoncé})}
($\bigstar$)
\label{subsec:corr6}
\begin{flushleft}
1)\\
\end{flushleft}

\subsection*{Exercice 7.}
\hyperref[subsec:ex7]{(\textit{Voir énoncé})}
($\bigstar$)
\label{subsec:corr7}
\begin{flushleft}
1)\\
\end{flushleft}

\end{document}
