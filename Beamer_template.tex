\documentclass[usenames,dvipsnames]{beamer} %for \cul xcolor tags to work with tikz, passes args to xcolor directly => DO NOT CALL the package with them later else causes error

\usepackage{Mmbase}


\usefonttheme[onlymath]{serif} %for serif font in math mode
\usetheme{Madrid}
\usecolortheme{default}
\setbeamertemplate{enumerate items}[default] %changes the dots of enumerate


\makeatletter
\newcommand{\Pause}[1][]{\unless\ifmeasuring@\relax
\pause[#1]%
\fi}
\makeatother %for correct \pause in math mode in enumerate



%------------------------------ Title page ------------------------------

\title[Intro Fourier]{Introduction à la théorie de Fourier}

\subtitle{Généralités}

\author{M. Waharte}

\institute[PPS]{Polytech Paris-Saclay}

\date[MAT342]{MAT 342 : Théorie de Fourier - 2022}

\titlegraphic { 
\begin{tikzpicture}[overlay,remember picture]
\node[right=10pt] at (current page.150){
    \includegraphics[height=1.3cm]{upsud.PNG}
};
\end{tikzpicture}
}


%------------------------------------------------------------
%beginning of each section and highlights the current section

\AtBeginSection[]{
  \begin{frame}
    \frametitle{Sommaire}
    \tableofcontents[currentsection]
  \end{frame}
}
%------------------------------------------------------------


\begin{document}

\frame{\titlepage}



% 1.----------------------------- ToC ----------------------------

\begin{frame}
\frametitle{Sommaire}
\tableofcontents
\end{frame}


% 2.---------------------------------------------------------

\section{Définition}


%3. ---------------------------------------------------------

\begin{frame}
\frametitle{Définition}

\begin{block}{Défintion}
Soit $f : \R^n \ra \C$ intégrable, on note sa transformée de Fourier $\hat{f}$ ( ou $\fF(f)$) :  $$\hat{f} : \xi\in\R^n \mapsto \int_\R f(x) e^{-2i\pi x\xi} \dx$$
\end{block}


\begin{itemize} %<1-> means 1st slide and after and <2> only 2nd slide
    \item<1-> On peut aussi le définir :\\ • $\hat{f}(\xi)\mapsto \int_\R f(x) e^{-x\xi}\dx$
    \item<2-> • $\hat{f}(\xi)\mapsto \frac{1}{(\sqrt{2\pi})^n}\int_\R f(x) e^{-x\xi}\dx$
\end{itemize}
(La seule différence sera les coefficient des formules)

\end{frame}


%4. ---------------------------------------------------------


\begin{frame}
\frametitle{Exemple}
\begin{exemple}
Calculez la transformée de Fourier de $\mathbbm{1}_{[a,b]}$ et de $x\mapsto e^{-|x|}$.\\
\end{exemple} \pause

\begin{equation*}
   \begin{split} 
        \hat{f}(\xi) &= \int_\R e^{-|x|} e^{-2i\pi x\xi}\,dx = \Pause \int_{\R^-} e^{x(1-2i\pi\xi)}dx\; + \; \int_{\R^+} e^{x(-1-2i\pi\xi)}dx \\
        &\cul[OliveGreen]{ = \frac{2}{1+4\pi\xi^2} }\,.
    \end{split}  
\end{equation*}

\end{frame}


%5. ---------------------------------------------------------

\section{Propriétés}


%6. ---------------------------------------------------------

\begin{frame}
\frametitle{Propriétés}

Soient la translation $\tau_y(f) : x\in\R^n \mapsto f(x-y)$ et $\sC_{\ra0}$ l'ensemble des fonctions continues convergeant vers $0$ en $\pm\,\infty$.

\begin{alertblock}{Propriétés}
Soient $f,g\in L^1(\R^n)$, $\lambda\in\R$ et $k\in\N$.

\begin{enumerate}
    \item Si $h : x\mapsto e^{2\pi i \lambda x}f(x)$, alors $h$ intégrable et $\hat{h}=\tau_\lambda f$.
    \item $\fF(\tau_\lambda f)(\xi) = e^{-2i\pi x\xi} \hat{f}(\xi)$.
    \item $\fF (f \ast g) = \hat{f}\hat{g}$.
    \item Si $\forall \alpha\in\N^n\; |\alpha|\leq k, (x\mapsto x^\alpha f(x))\in L^1(\R)$, alors $\hat{f}\in\sC^k$ et $\partial^\alpha f = \fF[(-2i\pi x)^\alpha f(x)]$.
    \item Si $f\in\sC^k$ et $\forall |\alpha|\leq k, \;\partial^\alpha f\in L^1$ et $\in \sC_{\ra 0}$ avec $k-1$, alors $\fF(\partial^\alpha f)(\xi) = (2i\pi\xi)^\alpha \hat{f}(\xi)$.
\end{enumerate}

\end{alertblock}

\end{frame}

%7. ----------------------------------------------------

\section{Transformée inverse}


\end{document}